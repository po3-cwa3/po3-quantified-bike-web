\section{Product}
\subsection{User stories}
\subsubsection{Comparing daily trips}
Sophia uses her bike and the BOSS on a daily basis. 
She bikes to class every day and clips the BOSS on and off her bike so it does not get stolen. 
On the BOSS website, she can easily review each trip she made in the calendar overview. 
She can get a general impression of how she did each day, and then she can compare her stats with a ranked list of the stats of her friends. 
This includes the farthest traveled distance, the longest bike trips (in time), the fastest speeds, etc.

\subsubsection{Sharing biking experiences with friends}
Luke likes to go on bike trips for fun and exercise. 
He sometimes doesn’t know where he will go, but he can easily find out where he’s been in the BOSS’s detail page on the website. 
When he sees a beautiful view, or just something interesting, he can easily take a picture on the built-in camera with a click of a button. 
He can also press a second button to mark certain points of interest. 
The points of interest are displayed with the photos on a map of his route that he can share with his friends on various social media sites (Facebook, Twitter, Google Plus,…) On top of the map, he can share various statistics of his trip (how bumpy the road was, how fast he went, how hot and humid it was) along with personal comments and ratings. 
If he wants, he can also view the statistics and comments of fellow cycling friends. 

\subsubsection{Fitness statistics}
Johnny likes biking for sport, and loves using the BOSS to record his data. 
He always keeps the BOSS on his bike, and only has to flick a switch to turn it on or off. 
After each ride, he goes to the BOSS website to check specific details of how he did, and can compare his performance with previous rides, made possible by the detailed compare view of the BOSS website. 
Furthermore, he can also compare his data with his competitive friends, by use of simple numbers (for the mean speed, travel distance and travel time), or with graphs (speed during a trip, measured temperature throughout the trip, heart rate, etc.) Finally, he can challenge his friends to beat his results via various social media sites.

\subsection{Product description}
The BOSS records several different types of data with external sensors that are either connected to the Raspberry Pi or the Arduino Nano. 
If an internet server is available, the data is sent to the server. If the connection is lost, the data can be stored locally.
It also has three input buttons: there is a switch to turn it on or off, a button to take a picture with a camera and a button to indicate a point of interest. 
The data from the last two are combined with recorded GPS data. Exterior LEDs indicate whether the connection to the server is working or not.
The website starts with the user’s own dashboard, showing random statistics of them and their friends. 
Here they can either choose to go to their personal (calendar) view, or compare and share with their friends in the social view. 
The calendar view gives the user an overview of their rides on each day, and a more detailed view per day after the user requests it. 
They can also compare multiple days by selecting the chosen days in the calendar view. 
The social view allows the user to compare his statistics with a friend, or find multiple rankings based on different statistics (e.g. highest speed, most distance traveled, longest bike ride).
The user can also share his results and challenge friends on multiple social media sites.
