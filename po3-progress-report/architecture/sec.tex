\section{Architecture}
Our application mainly consists of 3 parts. 
A Raspberry Pi and an Arduino Nano are mounted on the bike to collect data. 
The last part is the web interface.
\subsection{Device}
\subsubsection{Arduino Nano}
\textit{The code for the Arduino can be found on Github: \url{https://github.com/po3-cwa3/po3-quantified-bike-device} in the folder 'arduino'.}

The Arduino Nano is connected to a GPS Sensor, a temperature sensor, a humidity sensor, a heart beat sensor and a few buttons and LEDs. 

The code on the Arduino interprets the data coming from the sensors and sends it via a serial interface to the Raspberry, which is connected via a USB cable.

The interval between consecutive data transmissions depends on the kind of sensor. 
For instance, the state of a push button is sent more frequently than the GPS data.
All data is sent in a specific format. One packet of sensor data consists of a single line.
A semicolon is used as delimiter to separate values. The first value of a data packet identifies the sensor. 
The format of the next values depends on the kind of sensor.
\subsubsection{Raspberry Pi}
\textit{The code for the Raspberry can be found on Github: \url{https://github.com/po3-cwa3/po3-quantified-bike-device} in the folder 'raspberry'.}

On the Raspberry Pi, a Python application reads data from the Arduino. 
When an internet connection is available, the application can send the data to the server in real-time.

Alternatively, all trip data can be stored in a local MySQL database. 
As soon as an internet connection is available, that data can be sent to the server by a separate script.
When that data transmission succeeds, the data is removed from the local database.
\subsection{Web interface}
\textit{The code for the website can be found on Github: \url{https://github.com/po3-cwa3/po3-quantified-bike-web}.}

\textit{The website itself can be found at: \url{http://jorestha.be/peno/}.}

For the web interface we use one central JavaScript class. This class manages all data queries from the database.
It also takes care of all necessary initialisations for the main webpage. This includes the calendar initialisation and
the basic actions initialisation (button actions etc.).

When data gets queried for the current month, the data is temporarily stored. The data
model uses an array of day objects. These objects all have two properties.
\begin{enumerate}
	\item \textit{\textbf{trips} (Array):} contains the trips for that day.
	\item \textit{\textbf{average} (Object):} contains the averages for that day.
	Examples are average speed and total distance.
\end{enumerate}
A properties array is used to identify the different average properties. This array
is used for the calendar bar property and the details page. For the details page this
array also holds the postfix and precision of every property. For total distance, the
postfix would be 'km' for example.
